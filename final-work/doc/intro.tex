\begin{section}{Introducción}
Los flujos son gobernados por ecuaciones diferenciales parciales, que representan las leyes de conservación de masa, momento y energía. La dinámica de fluidos computacional se encarga de resolver esas ecuaciones diferenciales utilizando técnicas de análisis numérico. Las computadoras son utilizadas para realizar los cálculos requeridos para simular la interacción entre líquidos, gases y superficies definidas por las condiciones de borde. Disponer de mas poder computacional es útil para disminuir el tiempo requerido para realizar las simulaciones, o aumentar la calidad de los resultados.

Para poder aumentar el poder computacional disponible, se utilizan a menudo, técnicas de computación en paralelo. La computación en paralelo consiste en la realización de varios cálculos, o la ejecución de varios procesos de forma simultanea. Para poder aprovechar esta forma de calculo, los problemas grandes deben ser divididos en problemas mas pequeños que puedan resolverse independientemente, o con el intercambio de solo pequeñas cantidades de información entre los agentes que resuelven cada parte.

Para facilitar el trabajo de la programación de una solución paralela, se utilizara en este trabajo MPI (message passing interface). MPI es un sistema de pasaje de parámetros estandarizado y portable, que puede ser utilizado en una gran variedad de arquitecturas paralelas, y tiene un uso muy difundido en el campo de la computación de alto rendimiento. 

En este trabajo se realizara una simulación de las ecuaciones de Navier Stokes bidimensionales. Concretamente se resolverá el problema del flujo interno en un contenedor rectangular, en cuyo centro se encuentra situada una hélice que perturba el fluido creando así un campo de velocidades. Esta simulación es de particular interés para aquellos que trabajen con dispositivos similares, sean estos reactores químicos, enfriadores, o torres de mezclado. 

\end{section}


